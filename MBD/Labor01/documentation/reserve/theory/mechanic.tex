\section{Nachführen des Wasserstrahls}

%Die Regelstrecke beinhaltet den mechanischen Teil zum Nachführen wie auch den Wasserstrahl. Einen kleinen Einfluss auf die Strecke dürfte auch das Ventil haben.
%\todo{Bild Übertragungsfunktion der Strecke/ Weg von Regelstrecke}


	\subsection{Bewertungskriterien der Nachführung}
		\begin{description}
			\item Totzeit des Wasserstrahls
			\item Kosten
			\item Aufwand
			\item Füllgeschwindigkeit
			\item Platzbedarf
			\item Coolness
			\item Spezielles 
			\item Lieferfristen
			
		\end{description}
	
	
	\subsection{Möglichkeit zum Nachführen der Düse}
		
		\subsubsection{Bogenstrahl}
		Diese Möglichkeit wurde in der Aufgabenstellung beschrieben. Die Düse hat einen festen Platz. Es wird lediglich ihr Winkel verändert um das Glas in einem Bogen zu treffen.
		
		\subsubsection{Nachführen auf der Ebene}
		Nach dem Vorbild eines 3D-Druckers soll bei dieser Variante die Düse in X- und Y-Achse bewegt werden. Bei Bedarf könnte man auch noch die Z-Achse mit einbeziehen.\\
		Anstelle von X und Y könnte man auch im Radialsystem arbeiten mit $\Phi$ und Radius.
		
		\subsubsection{Fläche mit Düsen}
		Anstelle die Düse zu Bewegen könnte man auch das komplette Areal mit Düsen abdecken. Hätte jede Düse ein eigenes Ventil käme man wohl auf über 100Ventile. Folgend eine Überschlagsrechnung mit einem Düsenabstand von 2cm. Eine Düse würde folglich $400m^2$ abdecken.
			\begin{equation}
				A = r^2 * \pi = \left( 200mm\right) ^2 * \pi = 62'832mm^2\\
				n = \frac{A_{Areal}}{A_{pro_Düse}} = \frac{62'832mm^2}{400mm^2} = 157
			\end{equation}
		Es gäbe noch die Variante, dass mehrere Ventile geöffnet werden müssen um bei einer Düse einen Wasserstrahl rauskommen zu lassen. Würden wir zum Beispiel die X- und Y- Achse einzeln ansteuern gäbe dies 30 Ventile bei gleichem Düsenabstand. \todo{Skizze}\\
		\todo{Optimieren der Anzahl....}
		
		\subsubsection{Drohne}
		Mit einer Flugdrohne hätte man wohl die grösste Flexibilität und könnte ohne grossen Aufwand ein viel grösseres Areal abdecken als vorgesehen. \\
		Würde man den Wassertank an der Drohne befestigen müsste diese vermutlich etwa 10kg Nutzlast fliegen können. Dies würde bedeuten, dass auch die Drohne dementsprechend viel Platz benötigt.\\
		Die andere Variante wäre die Drohne an eine "Leine" zu nehmen und das Wasser bei Bedarf über einen Schlauch hoch zu pumpen.
	
	
	\subsection{Bewertungen}
	
\section{Ventil}
\todo{Mögliche Ventile/ Wasserhähne raus suchen}
	
\section{Wassertank}


