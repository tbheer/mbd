\section{Untersuchungen}

%\subsection{Untersuchungen}
Mit den erzielten Resultaten ergeben sich weitere Fragestellungen,
welche in Folgearbeiten beantwortet werden sollten. Im Folgenden
soll zusammengefasst werden, welche Arbeiten und Untersuchungen
im Anschluss an die Arbeit durchzuführen sind.

\subsection{Alterungseinfluss auf die Modellparameter}
Die Arbeit stellt eine Möglichkeit vor, wie mit generischen
Modellparametern eine batteriespezifische Identifikation
umgangen werden kann. Der hierfür verwendete Stichprobenumfang,
auf welchem sich die getroffenen Annahmen stützen, ist mit
lediglich 10 Batterien sehr klein. Hierzu sind weitere
Batterien zu untersuchen, mit welchen für alle Batterietypen
belegt werden kann, dass der angenommene Alterungseinfluss
vorliegt.

\subsection{Einfluss des Ladens auf die SOC Schätzung}
Die Tests des realisierten Prototypen zeigen, dass die erzielten
\gls{soc} Schätzungen eine systematische Überschätzung während
der Erhaltungsladung zeigen. Die Validierung des 2-Puls Lasttest,
welche auch zur Ermittlung der Modellparameter diente, wurde ohne
gleichzeitiges Laden der Batterien durchgeführt. Um die
Modellparameter für den realen Einsatz mit gleichzeitiger
Ladung zu ermitteln, ist eine weitere Validierung unter diesen
Betriebsbedingungen durchzuführen. Alternativ ist eine Korrektur
der bereits angewendeten Modellparamer zu untersuchen, welche
empirisch aus den erzielten Testergebnissen zu bestimmen ist.

\subsection{Dauertest des Diagnosesystems}
Der realisierte Prototyp des Diagnosesystems ist gemäss dem
entwickelten Vorschlag fertigzustellen und einem Dauertest
zu unterziehen. Hierbei sollte insbesondere untersucht werden,
wie sich die Ergebnisse der \gls{soh} Schätzung entwickeln.
Zur Überprüfung sollte eine regelmässige Referenzmessung
durchgeführt werden, wie diese in der vorliegenden Arbeit
vorgestellt wurde.

\subsection{Modellparameter für weitere Temperaturen}
Die durchgeführte Validierung des 2-Puls Lasttest untersuchte
das Verfahren für zwei Temperaturen, für welche auch die
entsprechenden Modellparameter ermittelt wurden. Für die
Realisierung der vorgeschlagenen Temperaturkompensation,
welche den gesamten thermischen Einsatzbereich der
Batterien berücksichtigt, sind weitere Messungen notwendig.

\subsection{Kompensation der Batteriekabel}
Aufgrund der Tatsache, dass in der vorliegenden Anwendung
unterschiedliche Installationen mit verschiedenen Batteriekabeln
verwendet werden, wurde in der vorliegenden Arbeit eine direkte
Messung an den Batterieanschlüssen realisiert. Für die reale
Anwendung würde dies höhere Kosten der Verkabelung bedeuten.
Um diese zusätzlichen Kosten zu umgehen, könnte eine Korrektur
mittels einer Modellparametrierung eingesetzt werden. Hierzu
müssen die unterschiedlichen Verkabelungsvarianten untersucht
werden.
