\begin{appendices}
	
	\section{Matlab Code}
	
\begin{lstlisting}[caption={Matlabcode},label=code]
	% Variables for MBD Unit2
	%1.1
	clear all;
	
	J_sc = 343;     %A/m2
	G = 1000;       %W/m2
	G_n = 1000;     %W/m2
	A = 0.01266;    %m2
	R_s = 1e-6;     %Ohm
	R_sh = 1000;    %Ohm
	I_0 = 2.8e-9;   %A
	Vt = 25.85e-3;  %V
	m = 1;          % 1= ideal diode
	V = 0.5;          %V
	k= 1.38064852*10^-23;
	T = 26.85 +273.15;
	q=1.602176565*10^-19;
	
	save var.mat;
	
	%%
	% Logdaten auslesen
	% logsout{2}.Values.Data
	% 2.4
	figure(99)
	subplot(2,2,1);
	plot(logsout{2}.Values.Time, logsout{2}.Values.Data)
	xlabel('Time [sec]')
	ylabel('Current [A]')
	
	subplot(2,2,2);
	plot(logsout{2}.Values.Time, logsout{4}.Values.Data ./logsout{2}.Values.Data)
	xlabel('Time [sec]')
	ylabel('Resistance [Ohm]')
	
	subplot(2,2,3);
	plot(logsout{2}.Values.Time, logsout{4}.Values.Data)
	xlabel('Time [sec]')
	ylabel('Voltage [V]')
	
	subplot(2,2,4);
	plot(logsout{2}.Values.Time, logsout{3}.Values.Data)
	xlabel('Time [sec]')
	ylabel('Power [W]')
	
	%saveas(gcf, '../../documentation/figures/subplots.eps', 'eps')
	
	%%
	% XY Plot with u & i
	% 2.4
	sim('unit2');
	x=u.get('Data');
	y=i.get('Data');
	plot(x,y);
	xlabel('Voltage [V]')
	ylabel('Amps [A]')
	
	
	%%
	%sim('unit2', 'G', [400, 600,800,1000,1200])
	% Simuliere mit verschiedenen G-Werten und speichere die Werte
	% Sim Zeit nehmen mit tic - toc
	% 3.1 & 4
	tic
	
	G = 400;
	sim('unit2');
	u400=u.get('Data');
	i400=i.get('Data');
	
	G = 600;
	sim('unit2');
	u600=u.get('Data');
	i600=i.get('Data');
	
	G = 800;
	sim('unit2');
	u800=u.get('Data');
	i800=i.get('Data');
	
	G = 1000;
	sim('unit2');
	u1000=u.get('Data');
	i1000=i.get('Data');
	
	G = 1200;
	sim('unit2');
	u1200=u.get('Data');
	i1200=i.get('Data');
	
	toc
	
	%%
	% Zeit um Array zu lesen
	tic
	[P_max, max_ind] = max(u1200.*i1200)
	toc
	
	%%
	%XY Plot Leistung über Spannung
	% 3.2
	plot(u400, i400.*u400, 'r')
	hold on
	xlabel('Voltage [V]');
	ylabel('Power [W]');
	plot(u600, i600.*u600, 'g')
	plot(u800, i800.*u800, 'b')
	plot(u1000, i1000.*u1000, 'k')
	plot(u1200, i1200.*u1200, 'c')
	legend({'G = 400', 'G = 600', 'G = 800', 'G = 1000', 'G = 1200'},'Location','southwest')
	hold off
	%saveas(gcf, '../../documentation/figures/xy_vp.eps', 'eps')
	
	%%
	%XY Plot Strom über Spannung
	% 3.2
	plot(u400, i400, 'r')
	hold on
	xlabel('Voltage [V]');
	ylabel('Current [A]');
	plot(u600, i600, 'g')
	plot(u800, i800, 'b')
	plot(u1000, i1000, 'k')
	plot(u1200, i1200, 'c')
	legend({'G = 400', 'G = 600', 'G = 800', 'G = 1000', 'G = 1200'},'Location', 'southwest')
	hold off
	%saveas(gcf, '../../documentation/figures/xy_vc.eps', 'eps')
	
	%%
	% XY Plot mit den max values in der Legende
	[P_value, index_peak] =  max(u1000.*i1000);
	figure(3);
	plot(u1200, i1200, 'r')
	xlabel('Voltage [V]');
	ylabel('Current [A]');
	legend(strcat('G = 1000, V_{mpp} = ', num2str(u1000(index_peak))));	
	
	%%
	% XY Plot mit den max values in der Legende
	% 3.3
	figure(2);
	plot(u1200, u1000.*i1000);
	xlabel('Voltage [V]');
	ylabel('Power [W]');
	legend(strcat('G = 1200, P_{mpp} = ', num2str(P_value)));
	
	%%
	%%Erstelle die Arrays zur Visualisierung für die Lookuptables mit undersampling
	% 4.0
	% i1arr = i1.get('Data');
	% u2arr = u1.get('Data');
	% 
	% c2 = 1;
	% for c = 1:8    
	%     p400(c) = u400(c2) * i400(c2);
	%     p600(c) = u600(c2) * i600(c2);
	%     p800(c) = u800(c2) * i800(c2);
	%     p1000(c) = u1000(c2) * i1000(c2);
	%     p1200(c) = u1200(c2) * i1200(c2);
	%     c2 = c2 + 40;
	% end
	
	% for c = 1:1000    
	%     iv(c) = i1arr(c2);
	%     uv(c) = u2arr(c2);
	%     c2 = c2 + 1;
	% end
	% 
	% %plt = [p400; p600; p800; p1000; p1200];
	% plt2 = [iv; uv];
	% toc
	% figure(4);
	% plot(uv, iv);
	% xlabel('Voltage [V]');
	% ylabel('Ampere [A]');
	% legend();
	% toc
	%%Erstelle den Sourcecode der Arrays für LaTex
	%latextable = latex(sym(vpa(round(plt2, 4))))
	%latextable = latex(sym(vpa(round(plt, 4))))
	
	%%
	% Simulation mit verschiedenen Temperaturen
	% 5.2
	
	T = 15 + 273.15;
	sim('unit2');
	u15=u.get('Data');
	i15=i.get('Data');
	
	T = 20 + 273.15;
	sim('unit2');
	u20=u.get('Data');
	i20=i.get('Data');
	
	T = 30 + 273.15;
	sim('unit2');
	u30=u.get('Data');
	i30=i.get('Data');
	
	T = 40 + 273.15;
	sim('unit2');
	u40=u.get('Data');
	i40=i.get('Data');
	
	T = 50 + 273.15;
	sim('unit2');
	u50=u.get('Data');
	i50=i.get('Data');
	
		
	%%
	%XY Plot Leistung über Voltage
	% 5.3
	plot(u15(1:end), i15(1:end).*u15(1:end), 'r')
	hold on
	xlabel('Voltage [V]');
	ylabel('Power [W]');
	plot(u20(1:end), i20(1:end).*u20(1:end), 'g')
	plot(u30(1:end), i30(1:end).*u30(1:end), 'b')
	plot(u40(1:end), i40(1:end).*u40(1:end), 'k')
	plot(u50(1:end), i50(1:end).*u50(1:end), 'c')

	legend({'T = 15C', 'T = 20C', 'T = 30C', 'T = 40C', 'T = 50C'},'Location','southwest')
	hold off
	%saveas(gcf, '../../documentation/figures/xy_vp_t.eps', 'eps')
	
	%%
	%XY Plot Strom über Spannung
	% 5.3
	plot(u15(1:end), i15(1:end), 'r')
	hold on
	xlabel('Voltage [V]');
	ylabel('Current [A]');
	plot(u20(1:end), i20(1:end), 'g')
	plot(u30(1:end), i30(1:end), 'b')
	plot(u40(1:end), i40(1:end), 'k')
	plot(u50(1:end), i50(1:end), 'c')
	legend({'T = 15C', 'T = 20C', 'T = 30C', 'T = 40C', 'T = 50C'},'Location', 'southwest')
	hold off
	%saveas(gcf, '../../documentation/figures/xy_vc_t.eps', 'eps'	
\end{lstlisting}
	
	
\end{appendices}